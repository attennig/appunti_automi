%!TEX root=../../root.tex

\section{Lezione 6}
\subsection{Pumping lemma}
$L \in REG \Rightarrow \exists n > 0$ t.c. $\forall w \in L$  $|w| \geq n$ $\exists x, y, z$  $w = xyz$\newline
Vale:
\begin{enumerate}
	\item $|xy| \leq n$
	\item $|y| \geq 1$
	\item $\forall i \geq 0 xy^iz \in L$
\end{enumerate}
\textbf{Dimostrazione:}
$L \in REG \Rightarrow \exists A \in DFA$ che riconosce $L$ con $n$ stati. Sia $p \in L$ t.c. $|p| \geq n$, il cammino che determina in A \`e composto da almeno $n+1$ stati. Se il cammino \`e composto da un numero di nodi maggiore del numero degli stati vuol dire che in esso \`e presente un ciclo. Inoltre possiamo dire che questo ciclo si presenta necessariamente entro l'n-esimo passo di computazione poich\'e nella sequenza di nodi attraversati ci sar\`a necessariamente almeno una ripetizione.\newline
Questo lemma negato \`e utile per dimostrare la non regolarit\`a di un linguaggio.

\subsection{Pumping lemma negato}
$L \notin REG \Rightarrow \forall n > 0 \exists w \in L$ $|w| \geq n$ $\forall x, y, z$ t.c. $w=xyz$  \newline
Vale:
\begin{enumerate}
	\item $|xy| \leq n$
	\item $|y| \geq 1$
	\item $\exists i \geq 0 xy^iz \notin L$
\end{enumerate}

\subsection{Applicazione Pumping lemma negato}
\subparagraph{Esempio 1:} Dimostrare che $L= \{ ww | w \in \{0,1\}^{\star}\} \notin REG$. \newline
Dato $p \in \mathbb{N}$ vogliamo trovare una parola $w \in L$ t.c. $|w| \geq p$ e che per ogni sua scomposizione $w=xyz$ esiste un $i$ t.c. $W'=xy^iz \notin L$. \newline
Scriviamo in maniera generica $w$. Ad esempio $w = 0^p10^p1$, questa parola \`e evidentemente pi\`u lunga di $p$. \newline
Scegliamo come scomposizione $x = 0^r$,$y=0^s$,$z=0^{p-(r+s)}10^p$ con $r \geq 0$, $s \> 0$, $t \geq 0$.\newline
Adesso scriviamo $w'=xy^iz$ per la scomposizione che abbiamo scelto:
$0^r(0^s)^i0^{p-(r+s)}10^p1$.\newline
Affinch\'e $w' \notin L$ deve essere vero che: $r+is+p-(r+s) \neq p$, questo perch\`e vogliamo "rompere" la simmetria della parola. Semplificando otteniamo che $is-s \neq 0$ e questo \`e vero $\forall i \neq 1$.\newline
Pi\`u intuitivamente si pu\`o dire che per "rompere" la simmetrie basta elevare a 0 y per ottenere un numero di 0 nella prima met\`a della parola inferiore al numero di 0 nella seconda met\`a della parola.
\subparagraph{Esempio 2:} Dimostrare che $L = \{w | w \in \{a,b\}^{\star} \land n_a(w)=n_b(w)\} \notin REG$. \newline
Dato $p \in \mathbb{N}$ vogliamo trovare una parola $w \in L$ t.c. $|w| \geq p$ e che per ogni sua scomposizione $w=xyz$ esiste un $i$ t.c. $w'=xy^iz \notin L$.\newline
Procediamo scrivendo in maniera generica una parola che appartiene ad L. Ad esempio:\newline
$w=b^p a^p$, $|w| \geq p$ dato che $|b^p| = p$.\newline
Ora scomponiamo in maniera generica $w$:\newline
$x = b^r$, $y = b^s$, $z= b^t a^p$ t.c. $r \geq 0$, $s \> 0$, $t \geq 0$ e $r+s+t = p$.\newline
Adesso scriviamo $w'=xy^iz$ per la scomposizione che abbiamo scelto:
$b^r (b^s)^i b^t a^p$.\newline
Per ottenere $w' \notin L$ deve valere che $r+is+t \neq p$, ma questo \`e vero $\forall i \neq 1$. 
\subparagraph{Esempio 3:} Dimostrare che $L'=\{w | w \in \{a,b\}^{\star}\land n_a(w) \neq n_b(w)\} \notin REG$.\newline
Sta volta il Pumping lemma non ci aiuta e dobbiamo ragionare diversamente. Nell'esempio 3 \`e stato dimostrato che $L \notin REG$ ma $L=L'^c$ quindi possiamo concludere che $L' \notin REG$ poiché il suo complemento non \`e regolare. 
\subparagraph{Esempio 4:} Dimostrare che $L = \{ a^nb^m | n \neq m \land n,m \geq 0 \} \notin REG$.\newline
Per questo esempio di nuovo il Pumping lemma non ci aiuta. Proviamo a dimostrare la sua non regolarit\`a per assurdo. \newline
Supponiamo che $L \in REG$ e consideriamo un altro linguaggio che possiamo dimostrare non essere regolare sfruttando il Pumping lemma: $L' = \{a^nb^m | n=m \land n,m \geq 0 \}$. Definiamo ora il complemento di $L'$ come unione di $L$ e di un altro linguaggio. Essendo $L'$ il linguaggio delle parole formate da un certo numero di a seguite dallo stesso numero di b il suo complemento sar\`a formato da:
\begin{itemize}
	\item $L$, ossia tutte le parole che seguono l'ordinamento di $L'$ (a prima di b) in cui il numero di a differisce dal numero di b;
	\item $\{xbyaz | x, y, z \in \{a,b\}^{\star} \}$, ossia tutte le parole che non seguono l'ordinamento di $L'$. 
\end{itemize}
Ora possiamo scrivere che $L'^c = L \cup \{xbyaz | x, y, z \in \{a,b\}^{\star} \}$. Abbiamo supposto che $L \in REG$, sappiamo che $\{xbyaz | x, y, z \in \{a,b\}^{\star} \} \in REG$. La classe dei linguaggi regolari \`e chiusa rispetto all'unione quindi $L'^c \in REG$, ed \`e chiusa anche rispetto al complemento quindi $L' \in REG$. Questo \`e un assurdo in quanto sappiamo che $L' \notin REG$.
