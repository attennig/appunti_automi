%!TEX root=../../root.tex

\section{Lezione 9}
\subsection{La classe dei linguaggi descrivibili dalle CFG}
Possiamo ora definire una nuova classe di linguaggi, ossia quelli denotati dalle grammatiche context free. 
\[
	L(CFG) = \{ L | \exists G \in CFG \land L(G) = L \}
\]
Come la classe dei linguaggi regolari, anche questa classe gode di alcune propriet\`a di chiusura rispetto a delle operazioni che ci consentono di creare grammatiche complesse sulla base di grammatiche semplici.
\begin{description}
	\item \emph{Unione:} Dati $L, L' \in L(CFG) \Rightarrow L \cup L' \in L(CFG)$ \newline Siano $G_1 = (V_1, \Sigma, R_1, S_1) \in CFG$ e $G_2= (V_2, \Sigma, R_2, S_2)$ definiamo l'unione delle due grammatiche come segue: $G_1 \cup G_2 = (V_1 \cup V_2, \Sigma, R_1 \cup R_2 \cup \{ S \to S_1 | S_2\}, S)$. Aggiungendo la regola $S \to S_1 | S_2$ stiamo imponendo che la prima derivazione sia la o la variabile di partenza di $G_1$ o quella di $G_2$. Notare che non vogliamo che le regole si mescolino per tale motivo deve valere che $V_1 \cap V_2 = \emptyset$.
	\item \emph{Prodotto:} Dati $L, L' \in L(CFG) \Rightarrow LL' \in L(CFG)$\newline Siano $G_1 = (V_1, \Sigma, R_1, S_1) \in CFG$ e $G_2= (V_2, \Sigma, R_2, S_2)$ definiamo il prodotto delle due grammatiche come segue: $G_1 G_2 = (V_1 \cup V_2, \Sigma, R_1 \cup R_2 \cup \{ S \to S_1S_2\}, S)$. Aggiungendo la regola $S \to S_1S_2$ stiamo imponendo che la prima derivazione sia la concatenazione delle variabili di partenza di $G_1$ e di $G_2$. Per lo stesso motivo di prima $V_1 \cap V_2 = \emptyset$. 
\end{description}
\subparagraph{Esempio:}  Dato il linguaggio $L = \{ 0^n1^m | n \neq m \land n, m \geq 0\}$ vogliamo trovare una grammatica per $L$ scomponendolo in linguaggi pi\`u semplici, per i quali trovare una grammatica sia facile, per poi unirle ed ottenere una grammatica per $L$. \newline Possiamo scrivere $L$ come l'unione dei seguenti linguaggi: 
\begin{description}
	\item $L_1 = \{0^n, n \geq 1\}$, per il quale possiamo definire la grammatica con le seguenti regole $R_1 = \{S_0 \to 0|0S_0\}$ con $S_0$ variabile di partenza.
	\item $L_2 = \{1^m, m \geq 1\}$, per il quale possiamo definire la grammatica con le seguenti regole $R_2 = \{S_1 \to 1|1S_1\}$ con $S_1$ variabile di partenza.
	\item $L_3 = \{0^n1^m,  n > m > 0\}$ che senza perdita di generalit\`a possiamo riscrivere come $L_3 = \{0^{m+p}1^m, m, p > 0\}$, per il quale possiamo definire la grammatica con le seguenti regole $R_3 = \{S_2 \to S_0S', S' \to 0S'1 | 01\}$ con $S_2$ variabile di partenza.
	\item $L_4 = \{0^n1^m  m > n > 0\}$ che senza perdita di generalit\`a possiamo riscrivere come $L_3 = \{0^n1^{n+p}, n, p > 0\}$, per il quale possiamo definire la grammatica con le seguenti regole $R_4 = \{S_3 \to S'S_1\}$ con $S_3$ variabile di partenza.
\end{description}
Notare che, sapendo di dover fare l'unione, alcune variabili delle grammatiche che descrivono i linguaggi semplici $L_i, i \in \{1,2,3,4\}$ sono definite in un solo linguaggio e poi riutilizzate negli altri, ad esempio $S_0$ o $S'$. \newline
Combinando le regole scritte per il linguaggi semplici otteniamo che la grammatica per $L$ \`e $G = (V, \Sigma, R, S)$ dove: 
\begin{description}
	\item $V = \{S, S_0, S_1, S_2, S_3, S'\}$;
	\item $\Sigma = \{0, 1\}$;
	\item $R = \{ S \to S_0 | S_1 | S_2 | S_3$, $S_0 \to 0|0S_0$, $S_1 \to 1|1S_1$, $S_2 \to S_0S'$, $S_3 \to S'S_1$, $S' \to 0S'1 | 01\}$.
\end{description}
\subsection{Grammatiche ambigue}
Intanto diciamo che una \emph{derivazione leftmost} o \emph{derivazione canonica a sinistra} \`e una derivazione in cui l'ordine di derivazione da precedenza alle variabili scritte pi\`u a sinistra.\newline
Una grammatica $G$ si dice \emph{ambigua} se esiste almeno una parola in $L(G)$ che pu\`o essere derivata con almeno due diverse derivazioni leftmost.\newline
Data una grammatica ambigua $G$, ottenere $G'$ non ambigua ed equivalente a $G$ non \`e algoritmicamente risolvibile.\newline
Esistono anche \emph{linguaggi intrinsecamente ambigui}, ossia non esiste grammatica non ambigua che li genera. Ad esempio $L = \{a^ib^jc^k | i=j o j=k, i, j, k \geq 0 \}$
\subparagraph{Gestire l'ambiguit\`a} Segue un esempio di come modificare una grammatica ambigua per il linguaggio delle espressioni aritmetiche per renderla non ambigua.\newline
Data la grammatica $G = (V, \Sigma, R, E)$ dove:
\begin{description}
	\item $V = \{E, I\}$;
	\item $\Sigma = \{a, b, (, ), +, \times \}$;
	\item $R = \{ E \to (E) | E + E | E \times E | I, I \to a | b\}$;	
\end{description}
Possiamo vedere che \`e possibile ricavare l'espressione $a + b \times b$ attraverso due diverse derivazioni leftmost:\newline
\begin{description}
	\item $E \Rightarrow E \times E \Rightarrow E + E \times E \Rightarrow a + E \times E \Rightarrow a + b \times E \Rightarrow a + b \times b $
	\item $E \Rightarrow E + E \Rightarrow a + E \Rightarrow a + E \times E \Rightarrow a + b \times E \Rightarrow a + b \times b $ 
\end{description}
Modifichiamo ora la grammatica in modo che non risulti pi\`u ambigua:
\begin{description}
	\item $V = \{E, T, F, I\}$, dove T la possiamo interpretare come la variabile dei termini e F la come la variabile dei fattori;
	\item $\Sigma = \{a, b, (, ), +, \times \}$;
	\item $R = \{ E \to T | E + T, T \to F | T \times F, F \to I | (E), I \to a | b\}$;	
\end{description}
\subsection{Problemi decidibili e indecidibili}
Un problema si dice \emph{decidibile} se esiste un algoritmo che lo risolve.\newline
Un problema si dice \emph{indecidibile} se non esiste un algoritmo che lo risolve. \newline
Seguono esempi di problemi:
\begin{description}
	\item \emph{Problema 1:} Disambiguazione di una grammatica \`e un problema indecidibile
	\item \emph{Problema 2:} Dati $A_1, A_2 \in DFA$ dire se $L(A_1) \subseteq L(A_2)$ \`e decidibile. S\`i, possiamo riformulare $L(A_1) \subseteq L(A_2)$ come $L(A_1) \cup L(A_2)^c = \emptyset$ che sappiamo essere un problema decidibile
	\item \emph{Problema 3:} Dati $A_1, A_2 \in DFA$ dire se $L(A_1) = L(A_2)$ \`e decidibile? S\`i, perch\`e $L(A_1) = L(A_2) \Leftrightarrow L(A_1) \subseteq L(A_2) \land L(A_2) \subseteq L(A_1)$. 
	\item \emph{Problema 4:} Dato $A \in DFA$ dire se $L(A) = \Sigma^{\star}$ \`e decidibile. S\`i e possiamo vedere il problema in due diversi modi. Il primo \`e ricondurlo all'equivalenza di due linguaggi. Il secondo \`e riformulare il problema come segue: $L(A)^c = \emptyset$.
	\item \emph{Problema 5:} Dati $A_1, A_2 \in DFA$ dire se esiste una parola $x$ t.c. $x \notin L(A_1) \land x \notin L(A_2)$. Anche questo problema \`e decidibile e pu\`o essere risolto in due diversi modi. Possiamo dire che $(L(A_1) \cup L(A_2))^c \neq \emptyset \Rightarrow \exists x$ t.c. $x \notin L(A_1) \land x \notin L(A_2)$. Oppure possiamo risolvere verificando se almeno uno dei linguaggi \`e uguale a $\Sigma^{\star}$.
\end{description}
